
\documentclass{cccg16}

\usepackage{amssymb, amsmath}
\usepackage{graphicx}
\usepackage{hyperref}
\usepackage[utf8]{inputenc}
\usepackage[T1]{fontenc}
\usepackage{lmodern}
\usepackage{hhline}
\usepackage{wrapfig}
\usepackage{subfig}
\usepackage{listings}
\usepackage{courier}

\DeclareMathOperator{\sign}{sign}
\DeclareMathOperator{\fma}{fma}
\DeclareMathOperator{\round}{round}
\def\Jack#1{{\bf [[#1]]}\ignorespaces}
%\def\Jack#1{\ignorespaces}%% uncomment to hide remarks

\lstset{basicstyle=\ttfamily}

%\title{Low Order Comparison of Line-Quadric Intercepts}
\title{On the Precision to Sort Line-Quadric Intersections}
\author{Michael Deakin \and Jack Snoeyink\thanks{School of Computer
    Science, University of North Carolina at Chapel Hill, {\tt
      mfdeakin@cs.unc.edu}, {\tt snoeyink@cs.unc.edu}}}

\index{Deakin, Michael}
\index{Snoeyink, Jack}

\begin{document}
\thispagestyle{empty}
\maketitle

\begin{abstract}
  We consider a subproblem of the task of exactly tracking a neutron
  moving along a given line segment through a CAD model with quadric
  surfaces, namely, how much more than the input precision is required
  to guarantee the correct ordering along the line segment of the
  intersections from a pair quadrics.  When the order of all but one
  pair of intersections is known, known se show that a resultant can
  resolve the comparison using only half the precision that may be
  required to eliminate radicals by repeated squaring, and we compare
  the time and accuracy of our technique with converting to extended
  precision to calculate roots.
  
%%% JSS: ordering and increasing are talking about procedures before their inputs and outputs are made clear
%% Ordering line-quadric intercepts at precisions greater than the input
%% precision by comparing the parameters $t_i$ corresponding to the
%% intercepts of the line can result in significant error.  Increasing
%% the precision the computations reduces this error, but significantly
%% increases the computational work required.

%% Given a line and a pair of quadric surfaces, we develop an algorithm
%% that correctly computes the order in which the line intersects the
%% quadric surfaces when there is one ambiguous difference in roots.
%% This method requires only half the precision required by the parameter
%% comparison method.
\end{abstract}

\section{Introduction}
\Jack{representations are important, so define your representations for lines and quadrics before using them in a formula.} 
A line-quadric intersection is a point~$p=(x, y, z)$ computable from
the parametric line equation $l(t)=(m_x t + c_x, m_y t + c_y, m_z t +
c_z)$ that satisfies the quadric surface equation~$Q(x, y, z)=q_{xx}
x^2 + q_{xy} xy + q_{xz} xz + q_x x + \dots + q_{zz} z^2 + q_{z} z +
q_c = 0$.  The parameter of the line specifying the point of
intersection can be computed by finding the roots of~$Q(l(t))=a t^2 -
2 b t + c$.  This is easiest to do with the quadratic equation, but\Jack{remember: distinguish between math and computation for any procedure. I thought you were talking math, but you just switched to computation with limited precision.}
this method has issues with correctness when performed at the same
precision as the coefficients of the input line and quadric surfaces.

Correctness in this case is defined for a particular set of inputs. \Jack{This first sentence adds nothing but the mistaken impression that Correctness has already been defined.  The next is easier if you've already mentioned representations above.  And we don't really want all things to be integers -- well scaled floating point is fine.  For now we can just say single, and assume that the exponents are close enough that making them equal does not lose bits.}
Specifically, a computation is said to be correct if for all inputs
which are integers when multiplied by $2^{-e}$ and are less $2^p$, the
rounded result of a series of additions, multiplications, divisions,
and square roots is exact.  The correctness of the computed parameter
can be ensured by increasing the precision of the intermediate
computations.  Because all inputs are integers, additions can usually
be computed exactly without any increase in precision.  Multiplication
requires the precision to be increased to the sum of the precisions of
the input values.
% This is terrible!! Fix it!!!
Correctness requires the results of expressions to be exactly
representable in the final format, so in general the use of divisions
or square roots disqualifies a method from being correct.

% Not as terrible
When the sign of a computation is all that's required, a single square
root is permissible, as comparing the radicand and the square of the
other term will determine the sign of the result.
% Need a more precise way of saying this...
When multiple square roots are present, determining the sign is made
harder because the minimum number of bits required to reasonably
approximate the difference is unknown \cite{demaine33open}.  Luckily,
correctness of the sign computation can still be guaranteed if the
maximum possible rounding error is less than the magnitude of the
result.

In this work, we compare three different methods for computing the
order of line-quadric intersections.  These methods are specifically
developed for the case where only one pair of roots has a difference
which is potentially overwhelmed by the rounding errors in the
computation.  This is a notable limitation, but scenes of quadric
surfaces which can have more than one pair of ambiguous roots are rare
and should be dealt with specifically.

\section{Methods}
Three line-quadric intersection comparsion methods are evaluated in
terms of correctness, precision requirements, and FLOPs required.
These methods include the ``Approximate Comparison Method'', the
``Parameter Comparison Method'', and the ``Resultant Method''.

\subsection{Approximate Comparison Method}
The approximate comparison method computes, for $i\in\{1, 2\}$,\Jack{Never separate math by commas only; always use words.  Doing so here let me define $i$ before using it, instead of after. Avoid frac inline.} the
roots~$r_i^\pm=({b_i\pm\sqrt{b_i^2-a_ic_i}})/{a_i}$
approximately by computing each operation.  The order of
the approximate roots can be calculated exactly as~$\sign(r_1^\pm-r_2^\pm)$.

Though this method will usually give correct orders, it clearly does not guarantee them, since intermediate computations, such as the square roots,  cannot be computed
exactly for all inputs.\Jack{Shortened what you had.}


\subsection{Parameter Comparison Method}\Jack{The roots are parameters, so it is better to call this Repeated Squaring Method -- more parallel with Resultant Method.}
This method computes~$\sign(r_1^\pm-r_2^\pm)$ with some algebraic
manipulations to eliminate division and potentially bad square root
operations.  We eliminate these operations by using the following
property:~$\sign(y)=\sign(x)\sign(x\cdot y)$.  Divisions can be
removed by applying this directly, in this case bringing us
to~$\sign(a_1 a_2)\sign(a_1 a_2 (r_1^\pm-r_2^\pm))$.  One of the
square roots can be replaced by multiplying by $r_1^\pm-r_2^\mp$,
bringing us to~$\sign(r_1^\pm-r_2^\mp)\sign(a_1 a_2 (r_1^\pm -
r_2^\pm) (r_1^\pm - r_2^\mp))$.  When simplified, the final sign is
computed from~$2a_2^2b_1^2-4a_1a_2^2c_1+4a_1^2a_2c_2-2a_1a_2b_1b_2\pm
2 \sqrt{(a_1a_2b_2-a_2^2b_1)^2(b_1^2-4a_1c_1)}$. This is sufficient to
compute the correct result, as the simplified equation contains only
one square root, multiplications, and additions.

Despite the work done, it's not immediately clear that this leads to a
correctly computable method.  The remaining issues are present in the
other sign computation with square roots. $\sign(r_1^\pm-r_2^\mp)$ is
not correctly computable in general.  However, because we are
computing $\sign(r_1^\pm-r_2^\pm)$, we know that $r_1^\pm, r_2^\pm$
are the only pair roots for which the sign of the difference cannot be
computed exactly.  This resolves all issues with guaranteeing
correctness for this computation.

Though this method is correct, it requires a large increase in
precision to correctly compute the sign. The terms outside of the
square root can be computed correctly by increasing the precision
to~$4\times$ the coefficients precision.  The precision increases
to~$8\times$ after squaring for comparison to the radicand.  The terms
inside the radicand also require~$8\times$ the coefficients precision
to be computed correctly.

\subsection{Resultant Method}
Another method for computing the order of two intersections involves
computing the resultant for their respective polynomials.  The
resultant of two polynomials can be computed by taking the determinant
of the Sylvester Matrix.  The Sylvester Matrix for
polynomials~$P(t)=p_m t^m + \dots + p_0$ and~$Q(t)=q_n t^n + \dots +
q_0$ is defined in Equation \ref{eq:sylv}.
\begin{equation}
  S_{PQ}=\begin{pmatrix}
    p_m & \dots & & p_0 & 0 & & 0\\
    0 & p_m & \dots & & p_0 & & 0\\
    & \ddots & & & & \ddots\\
    0 & 0 & & p_m & \dots & & p_0\\
    q_n & \dots & & q_0 & 0 & & 0\\
    0 & q_n & \dots & & q_0 & & 0\\
    & \ddots & & & & \ddots\\
    0 & & q_n & \dots & & q_0\\
  \end{pmatrix}
  \label{eq:sylv}
\end{equation}
where~$P$ and~$Q$ are polynomials of one parameter and with orders~$m$
and~$n$, respectively \cite[Section~3.5]{cheeyap}.

The determinant of the Sylvester matrix is defined to be the resultant
of the two polynomials.  Given the roots~$a_i, b_i$ of~$P$ and~$Q$,
respectively, the resultant can also be computed as shown in Equation
\ref{eq:resultant}. \cite[Section~6.4]{cheeyap}.
\begin{equation}
  res(P, Q)=p_m^n q_n^m \prod_{i=1}^m\prod_{j=1}^n (a_i-b_j)
  \label{eq:resultant}
\end{equation}

The two methods of computing the resultant provides us with an extra
method of computing the sign of one of the differences of the two
roots.  Since we are concerned only with the signs of the differences,
the actual value doesn't matter.  Thus, if one of the differences is
too close to zero for comfort, we can compute the correct sign for it
from the signs of the other differences and the sign of the
determinant.  Equation \ref{eq:resultant} lets us write Equation
\ref{eq:signresult}.

\begin{equation}
  \sign(res(P, Q)) =
  \sign(p_m^n)\sign(q_n^m)\prod_{i=2}^m\prod_{j=1}^n[\sign(a_i-b_j)]
  \label{eq:signresult}
\end{equation}

Without loss of generality, let us be concerned with the sign
of~$a_1-b_1$.  The sign of~$a_1-b_1$ can then be computed as shown in
Equation \ref{eq:signroot}

\begin{figure*}
  \begin{align}
    \sign(a_1-b_1)=\sign(res(P, Q))\sign(p_m^n)\sign(q_n^m)
    \prod_{i=2}^m\prod_{j=2}^n[\sign(a_i-b_j)\sign(a_1-b_j)\sign(a_i-b_1)]
    \label{eq:signroot}
  \end{align}
\end{figure*}

For simplicity of computation, instead of performing many
multiplications, we need to count only the number of negatives on the
right hand side of this equation.  If there are an even number of
negatives, then the undetermined sign will be positive.  If there are
an odd number of negatives, the undetermined sign will be negative.
Since we are concerned with the intersections of lines with quadric
surfaces, we know our polynomials will be of order~$2$, and thus that
the signs of~$p_2^2$ and~$q_2^2$ will be positive and can be ignored.

It should be immediately apparent that computing the determinant can
be done more cheaply than correctly computing the sign of the
differences of roots of the polynomials.  Computing the roots of the
polynomials requires seven additions and multiplications, plus the
expensive square root and two divisions.  For reference, it has been
shown that a square root costs approximately the same number of
operations as~$\frac{3}{2}$ additions at the same precision
\cite{karatsuba}.  To compute the roots, two multiplications and an
addition are computed at 6 times the precision of the input.  The
square root and two additions are computed at 12 times the initial
precision.  The two divisions are computed at 24 times the initial
precision.  To actually perform the comparison, one final subtraction
is required at 24 times the initial precision. This means each
comparison requires only 1 FLOP, with an initialization cost of 10
FLOPs per intersection.

On the other hand, computing a determinant of a 4x4 matrix costs about
120 multiplications.  This is clearly too expensive, so we specialize
for the Sylvester matrix.  Computing the determinant of the Sylvester
matrix itself would naively take~$35$ FLOPs for each comparison.  We
can do better by computing the per-polynomial values shown in Equation
\ref{eq:sylvpoly}.  Since these values are solely determined by one of
the polynomial's coefficients, we compute these once at first use, and
store them for whenever we need to make future comparisons.  This
brings us to 11 FLOPs per comparison, with an initialization cost of 7
FLOPs per intersection.  This appears more expensive, but the
reduction in precision requirements and the lack of expensive square
roots and divisions results in significant savings.
\begin{figure*}
  \begin{equation*}
    \Delta=\begin{vmatrix}
    a_1 & b_1 & c_1 & 0\\
    0 & a_1 & b_1 & c_1\\
    a_2 & b_2 & c_2 & 0\\
    0 & a_2 & b_2 & c_2\\
    \end{vmatrix}=
    a_1^2 c_2^2 + c_1^2 a_2^2 + b_1^2 a_2 c_2 + b_2^2 a_1 c_1 -
    b_1 c_1 a_2 b_2 - a_1 b_1 b_2 c_2 - 2 a_1 c_1 a_2 c_2
  \end{equation*}
  \begin{align}
    \alpha_i=a_i^2,\,\, \gamma_i=c_i^2,\,\,
    \delta_i=a_i b_i,\,\, \epsilon_i=a_i c_i,\,\, \zeta_i=b_i c_i,\,\,
    D_i=b_i^2-\epsilon_i,\,\,
    i\in {1, 2}\\
    \Delta = \alpha_1 \gamma_2 + \gamma_1 \alpha_2 +
    D_1 \epsilon_2 + \epsilon_1 D_2 - \zeta_1 \delta_2 -
    \delta_1 \zeta_2
  \label{eq:sylvpoly}
  \end{align}
\end{figure*}
% CITE/Proof needed!!!
This reduction theoretically makes this method slightly cheaper than
computing the roots of the polynomial at higher precision.  Computing
the determinant this way always requires six multiplications and five
additions at 12 times the precision of the input.  Computing the
per-polynomial values requires six multiplications and one subtraction
at 6 times the precision of the input.

Since we are concerned only with the signs of the differences of
roots, no evaluation of the product of the known differences of the
roots is needed.  Thus, no other increase of precision is required for
the computation of the sign of the difference of the final pair of
roots.

\subsection{Evaluation of the Resultant Method}
The increased precision and resultant methods were implemented in a
C++ program for evaluation.  MPFR was used to implement the increased
precision in both cases.

To ensure the results from both methods were consistent, the ordered
lists of roots were divided into regions based on a relative threshold
heuristic.  This heuristic put an upper limit on the largest bit that
could change in a single region.  To compensate for gradual changes in
a region, this heuristic depended only on the largest root seen in the
region to this point.  If the number of roots in the region was
different for the two methods, the results are listed as possibly
being erroneous.  Note that this method does not verify that the
regions contain the same quadrics; as the quadrics in each region can
vary based on the order of previously seen quadrics.

To compare the methods, several scenes of quadric surfaces were
created.  These scenes were composed of surfaces which were mostly
constrained within the unit cube centered at~$(0.5, 0.5, 0.5)$.  To
actually perform the tests, 10000 random lines were generated.  These
lines were generated with intercepts positioned to make close
intersections more likely.  For most scenes, this was done by
constraining the intercept within the unit cube.  For the case of the
intersection of two orthogonal cylinders, this was positioned inside
of the two cylinders near the plane containing their intersection
curve.  In specific scenes like this, the direction of the line was
also constrained to further increase the probability of two
intersections being computed in the wrong order.

Once the line was generated, the intersections and their order were
computed along with some statistics.  These statistics were the amount
of time the sorting process took, the number of comparisons made at
higher precision, and whether the results of the two algorithms
approximately matched.  The C++ STL sort function was used to sort the
computed list of intersections, and the POSIX clock\_gettime function
was used for timing.  This is not an entirely fair comparison, for
several reasons.

The first is that the naive parameter comparison method would always
compute the increased precision result, regardless of whether it was
necessary.  This is unfair because it avoids the step of determining
whether the increase was actually necessary, which would be performed
in a real system.  In our testing setup, the threshold used to decide
whether the increased precision is necessary was set to infinity,
meaning the precision will always be increased in the resultant method
as well.  This skews the results slightly in favor of the naive
method.

The second issue with the comparison is that the resultant method
cannot always be used.  The resultant method can be used only when two
polynomials do not share roots.  It also requires that only one sign
be ambiguous.  Furthermore, though the resultant can be used to
compute the order of intersections for an arbitrary number of roots,
we have only implemented it for the case of~$2$ (possibly repeated)
roots for both polynomials.  This means that the line must intersect
the surface twice for this method to be usable.

For the sake of performance comparisons, the first two limitations on
the resultant method can be ignored.  This is because our tests here
are for speed more than accuracy.  The third issue cannot be ignored,
but can be avoided in the construction of our test scenes.  Any scene
which is used to evaluate the speed of the resultant method in our
test must ensure a line will either intersect a surface twice, or
intersect it once with a multiplicity of two.  This is guaranteed for
most surfaces, but can occur on rare occasion with parabolic
cylinders, hyperbolic paraboloids, and elliptic paraboloids.  To
prevent this issue from occurring, we do not include these types of
quadric surfaces in our test scenes.  We also do not include any
linear planes in our scenes.

To analyze the results, we computed the least squares linear fit of
the timing data.  For the Test Cylinders and Packed Ellipsoids, we
used the number of comparisons made as our independent variable, and
computed a linear coefficient for it along with a constant term.  For
the sets of Centered Ellipsoids and Aligned Cylinders, we was able to
aggregate the test results for each type of scene.  This allowed us to
fit the line to two independent variables; the number of comparisons
made, and the number of quadric surfaces in the scene.  Fitting the
time it takes to make a comparison to a linear relationship makes
sense, as the bulk of the sorting time is taken up by performing these
comparisons.  It also makes sense for the time to be linearly related
to the number of quadrics in the scene, as the cached values are only
computed once for each quadric, and are likely to significantly affect
the time a comparison takes.

\section{Test Hardware}
Three computers were used to test the implementations.  The first
computer has a Core i3 M370 processor with 2 cores and a 3 MB cache.
This processor does provide not a hardware implementation of FMA.
This computer has 4 GB of DDR3 memory clocked at 1 GHz.  It is running
an up to date installation of Arch Linux with version 4.4 of the
kernel.  GCC 5.3 was used to compile the code for these tests.  For
these tests, the performance manager was set to keep the CPU clock at
2.4 GHz, and the process was run with a nice value of -20.

The second computer has two Xeon E5-2643 processors with 4 cores and
10 MB caches.  This processor does not provide a hardware
implementation of FMA.  This computer has 32 GB of DDR3 memory running
with a 1.6 GHz clock.  It is running Ubuntu 14.04 with version 3.13 of
the kernel, and uses GCC 5.2 to compile the code for these tests.
These tests were run with the default performance manager with a
maximum CPU clock of 3.3 GHz, along with a nice value of -20.

The third computer has a Core 2 Duo E6550 processor with two cores and
a 4 MB cache.  This processor does not provide a hardware
implementation of FMA.  This computer has 8 GB of DDR2 memory clocked
at 667 MHz.  It is running an up to date installation of Gentoo Linux
with version 4.0 of the kernel.  GCC 5.3 was used to compile the code
for these tests.  For these tests, the performance manager was set to
keep the CPU clock at 2.3 GHz, with a nice value of -20.

To get a better estimate of the relative performance of the two
computers, the Geekbench benchmark was employed to estimate the
processor speeds.  The Ubuntu computer had a single core floating
point score of 2730.  The Arch computer had a single core floating
point score of 1702.  The Gentoo computer had a single core floating
point score of 1408.  On average, the Ubuntu computer was capable of
about 1.6 times more FLOPS than the Arch computer, and 1.9 times more
FLOPS than the Gentoo computer.

\section{Results}
\section{Conclusion}

\bibliographystyle{plain}
\bibliography{resultantmethod}
\end{document}
